%%%%%%%%%%%%%%%%%%%%%%%%%%%%%%%%%%%%%%%%%
% Professional Table
% LaTeX Template
% Version 1.0 (11/10/12)
%
% This template has been downloaded from:
% http://www.LaTeXTemplates.com
%
% License:
% CC BY-NC-SA 3.0 (http://creativecommons.org/licenses/by-nc-sa/3.0/)
%
% Note: to use this table in another LaTeX document, you will need to copy
% the \usepackage{booktabs} line to the new document and paste it before 
% \begin{document}. The table itself can then be pasted anywhere in the new
% document.
%
%%%%%%%%%%%%%%%%%%%%%%%%%%%%%%%%%%%%%%%%%

\documentclass{article}

\usepackage{booktabs} % Allows the use of \toprule, \midrule and \bottomrule in tables for horizontal lines

\begin{document}

\begin{table} % Add the following just after the closing bracket on this line to specify a position for the table on the page: [h], [t], [b] or [p] - these mean: here, top, bottom and on a separate page, respectively
\centering % Centers the table on the page, comment out to left-justify
\begin{tabular}{l c c c c c} % The final bracket specifies the number of columns in the table along with left and right borders which are specified using vertical bars (|); each column can be left, right or center-justified using l, r or c. To specify a precise width, use p{width}, e.g. p{5cm}
\toprule % Top horizontal line
& \multicolumn{5}{c}{Growth Media} \\ % Amalgamating several columns into one cell is done using the \multicolumn command as seen on this line
\cmidrule(l){2-6} % Horizontal line spanning less than the full width of the table - you can add (r) or (l) just before the opening curly bracket to shorten the rule on the left or right side
Strain & 1 & 2 & 3 & 4 & 5\\ % Column names row
\midrule % In-table horizontal line  
Cancer & 120 & 114 & 95 (90.8-98.3) & 90 & 75 (67.5-82.5) \\
Advanced Adenoma & 109 & 64 & 58.7 (49.5-67.9) & 21 & 19.3 (11.9-26.6) \\
Nonadvanced Adenoma & 89 & 49 & 55.1 (43.8-65.2) & 10 & 11.2 (5.62-18) \\
Any Lesion & 318 & 227 & 71.7 (66.4-76.4) & 121 & 38.1 (32.7-43.1)  \\
\midrule % In-table horizontal line
\midrule % In-table horizontal line
Negative Colonosopies & 172 & 144 & 83.7 (78.5-89) & 167 & 97.1 (94.2-99.4) \\ 
\bottomrule % Bottom horizontal line
\end{tabular}
\caption{Table caption text} % Table caption, can be commented out if no caption is required
\label{tab:template} % A label for referencing this table elsewhere, references are used in text as \ref{label}
\end{table}

\end{document}